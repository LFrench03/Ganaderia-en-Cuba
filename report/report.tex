\documentclass{article}
\usepackage[utf8]{inputenc}
\usepackage[hidelinks]{hyperref}
\usepackage{xcolor}
\usepackage[T1]{fontenc}
\usepackage{listings}
\usepackage{xcolor}
\usepackage{graphicx}
\usepackage[a4paper, left = 3cm, right = 3cm, top =2cm]{geometry}
\definecolor{codegreen}{rgb}{0,0.6,0}
\definecolor{codegray}{rgb}{0.5,0.5,0.5}
\definecolor{codepurple}{rgb}{0.58,0,0.82}
\definecolor{backcolour}{rgb}{0.95,0.95,0.92}
\definecolor{redflag}{rgb}{0.266,0,0}
\definecolor{blueflagg}{rgb}{0,87,214}
\definecolor{darkblueflag}{rgb}{0,33,66}
\definecolor{customGreen}{HTML}{228B22}
\definecolor{marron}{HTML}{804000}
\definecolor{orange}{HTML}{FFA500}
\lstdefinestyle{mystyle}{
    backgroundcolor=\color{backcolour},   
    commentstyle=\color{codegreen},
    keywordstyle=\color{magenta},
    numberstyle=\tiny\color{codegray},
    stringstyle=\color{codepurple},
    basicstyle=\ttfamily\footnotesize,
    breakatwhitespace=false,         
    breaklines=true,                 
    captionpos=b,                    
    keepspaces=true,                 
    numbers=left,                    
    numbersep=5pt,                  
    showspaces=false,                
    showstringspaces=false,
    showtabs=false,                  
    tabsize=2
}
\lstset{keepspaces=true, style=mystyle}

\title{\textcolor{crimson}{\Large{\textbf{Proyecto Final de Análisis Exploratorio de Datos}}} \\ \normalsize{\textit{Facultad de Matemática y Computación}} \\\normalsize{\textit{Ciencia de Datos \\ Grupo D111}}\\ \textcolor{customGreen}{\large{\textit{Dataset: Trees.}}}} 
\author{\textcolor{marron}{\normalsize{Guillermo Cepero García}} \\ \textcolor{marron}{\normalsize{Luis Ernesto Serras Rimada}} \\ \textcolor{marron}{\normalsize{Miguel Vadim Vilariño Pedraza}}}
\date{\today}
\begin{document}
\begin{titlepage}
    \begin{center}
    {\includegraphics[width=0.15\textwidth]{matcom.jpg}\par}
    \vspace{0.1cm}
    {\bfseries\LARGE Ciencia de Datos \par}
    \vspace{0.2cm}
    {\scshape\Large  Facultad de Matamática y Computación\par}
    \vspace{0.6cm}
    {\scshape\Huge{{\textbf{Comunicación de Ciencia de Datos \\ Proyecto Final}} \par}
    \vspace{0.2cm}
    {\itshape\Large {\large{\textit{La Ganadería en Cuba.}}} \par}
    \vspace{0.8cm}
    {\large Integrantes: \\ \normalsize{Luis Ernesto Serras Rimada} \\ \normalsize{Yulia Karla Felipe Quintana}  \par}
    \vspace{0.5cm}
    {\includegraphics[width=0.85\textwidth]{Identificador_principal.png}}
    \end{center}
\end{titlepage}
\newpage
\tableofcontents
\lstlistoflistings
\newpage
A continuación, se presenta el código en \textcolor{blue}{R} que permite calcular estas medidas y se explican brevemente cada una de ellas.

\begin{lstlisting}[language=R, caption=Medidas]
# Cargar librerias necesarias
library(e1071)  
# Funcion para calcular medidas estadisticas
medidas <- function(x) {
c(Media = mean(x),# Media aritmetica
Mediana = median(x),# Valor central
Moda = as.numeric(names(sort(table(x), decreasing = TRUE)[1])),# Valor mas frecuente
Varianza = var(x),# Dispersion de los datos
Desviacion_Estandar = sd(x),# Raiz cuadrada de la varianza
Rango = diff(range(x)),# Diferencia entre el valor maximo y minimo
Maximo = max(x),# Valor maximo
Minimo = min(x),# Valor minimo
Coeficiente_Variacion = sd(x) / mean(x),# Relacion entre la desviacion estandar y la media
Simetria = skewness(x),# Medida de asimetria de la distribucion
Curtosis = kurtosis(x))}# Medida de la "altura" de la distribucion  
# Aplicar la funcion a cada variable del dataset trees
resultados <- sapply(trees, medidas)
# Mostrar resultados
print(resultados)
------------------------------------------------------------------------------------
OUTPUT:
                            Girth      Height      Volume
Media                    13.2483871 76.00000000  30.1709677
Mediana                  12.9000000 76.00000000  24.2000000
Moda                     11.0000000 80.00000000  10.3000000
Varianza                  9.8479140 40.60000000 270.2027957
Desviacion_Estandar       3.1381386  6.37181293  16.4378464
Rango                    12.3000000 24.00000000  66.8000000
Maximo                   20.6000000 87.00000000  77.0000000
Minimo                    8.3000000 63.00000000  10.2000000
Coeficiente_Variacion     0.2368695  0.08383964   0.5448233
Simetria                  0.5010559 -0.35687727   1.0132739
Curtosis                 -0.7109412 -0.72336766   0.2460393
\end{lstlisting}

\end{document}